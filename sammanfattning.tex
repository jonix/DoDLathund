\documentclass[a4paper, 10pt, titlepage]{article}
\usepackage[utf8]{inputenc}
\usepackage[T1]{fontenc}
\usepackage{a4wide}              % Wide paper
\usepackage{parskip}


% For Swedish reports
\usepackage[swedish]{babel}
%\frenchspacing

% For encapsulated postscript figures
%
% Use:
%
% \includegraphics{width=10cm, height=10cm}{fig.eps}
%
% where width and height are optional
%\usepackage{graphicx}   % For eps figures

% Exchange text in encapsulated postscript figures with LaTeX text
%
% Before the image, insert arbitrary many
% (Don't worry, it _will_ look good when you convert to postscript)
%
% \psfrag{original text}{substituted text}
%
%\usepackage{psfrag}

%
% For nicer captions
%
% Valid options (between []) are:
%
% Indentation: hang, center, centerlast, nooneline
% Size: scriptsize, small, normalsize, large, Large
% Style: up, it, sl, sc, md, bf, rm, sf, tt
%
%\usepackage[hang,small,bf]{caption}

%
% Want to change font?
%
% Uncomment your choice, if all uncommented, Computer Modern Roman is
% used. Note that some of these don't seem to work properly
%
\usepackage{palatino}
%\usepackage{times}
%\usepackage{charter}
%\usepackage{utopia}
%\usepackage{pifont}
%\usepackage{chancery}
%\usepackage{bookman}
%\usepackage{avant}
%\usepackage{helvet}
%\usepackage{zapfchan}
%\usepackage{courier}
%\usepackage{newcent}

%
% Some optional packages:
%

%
% Fancier enumeration
% You get a new \begin{enumerate}[XXX] where you can specify XXX to be
% text {i,I,a,A,1}, for example \begin{enumerate}[Uppg. a)] to get a
% Uppg. a)/b)/c) list.
%
\usepackage{enumerate}


% And here the document begins!
\begin{document}

\paragraph{Drakar och Demoner - Ordlista}
\begin{enumerate}[I]
\item Rollperson $\Rightarrow$ RP (Den som är äventyraren)
\item Spelledare $\Rightarrow$ SL (Den som leder och skapar äventyren)
\item Spelledarperson $\Rightarrow$ SLP (En eller flera karaktärer som leds av Spelledaren)
\item Färdighetsvärde $\Rightarrow$ FV
\item (Magi) Skicklighetsvärde $\Rightarrow$ S (FV för en viss magisk besvärjelse)
\item (Magi) Skolvärde $\Rightarrow$ SP (Nivån på magisk mognad som måste uppfyllas för besvärjelser)
\item Erfarenhetspoäng $\Rightarrow$ EP (Sätt att förbättra sin karaktär efter äventyr)
\item Bakgrundspoäng $\Rightarrow$ BP (Poäng att använda för att skapa sin RP)
\item Chanser att lyckas $\Rightarrow$ CL (Ett modifierat FV vid mindre optimala situationer)
\item Stridsrunda $\Rightarrow$ SR (En handling vid strid, tar 5 sekunder)
\item Kroppspoäng $\Rightarrow$ KP (Din hälsa, finns Totala KP och KP för kroppsdelar)
\item Smärtpoäng $\Rightarrow$ SP (Trubbiga vapen ger mer smärta än skada)
\end{enumerate}

\paragraph{Drakar och Demoner - Lathund}

\begin{enumerate}[I]
\item Alla färdigheter är baserade på grundegenskaper
\item Grundegenskaper kan ge gratis Färdighetsvärde när Rollpersonen skapas
\item Alla Rollpersoner har samma primära färdigheter, dock är de olika duktiga
\item Primära färdigheter har ingen övre gräns för färdighetsvärden
\item Sekundära färdigheter begränsas av värdet för den associerade grundegenskapen
\item Rollpersoner kan inte lära sig sekundära färdigheter vid skapandet av rollpersonen
\item Vissa sekundära färdigheter kan bli yrkesfärdigheter. Vilka färdigheter beror på yrket
\item Rollpersonen får automatiskt F.V. i sina yrkesfärdigheter vid skapandet
\item Lättare att lära sig yrkesfärdigheter än sekundära färdigheter (Billigare)
\item Yrkesfärdigheter begränsas inte av den associerade grundegenskaperna värden
\item Magiker, Utbygdsjägare och Paladins är exklusiva i att lära sig magi
\item Bara magiker kan lära sig besvärjelser från början
\item En magiker (och andra) måste ha tillräcklig högt Skolvärde i magi-skolan för att lära sig besvärjelser
\item Skicklighetsvärde i besvärjelser är obegränsade (obegränsade av besvärjelsers skolvärde)
\item Besvärjelser är varken primära eller sekundära färdigheter
\item Besvärjelsers skicklighetsvärden kan jämnställas med vanliga färdighetsvärden
\item Låt inte regler diktera dig, i diskussion med SL kan de ändras. Låt de vara riktlinjer istället
\end{enumerate}

\section{Skapa Rollpersonen}
Fundera ett tag över vilken bakgrund din Rollperson har. Varför är han/hon just här, vad har hänt i livet som
gör att du äventyrar.

För en ordinär äventyrare så har du 125 bakgrundspoäng, för extraordinär har du 150 och en hjälte har 175.
För dessa bakgrundspoäng köper du det som gör din äventyrare unik.

Köpa ras, grundegenskaper, skaffa särskilda förmågor, tvåhänthet, socialt stånd, extra startkapital,
extra färdighetsvärden.

Det enda kravet är att köpa en ras och grundegenskaper

\section{Köpa Ras}
\paragraph{Alv} En alv har kattögon, spetsiga öron. Alver är mer spensliga och klengt byggda än människor,
å andra sidan har de en högre smidighet och är lite vackrare. De är lite intelligentare och har bättre
hörsel och syn än en människa. En Alv kan inte dö av ålder.

En Alv får +4 FV i färdigheterna Lyssna och Upptäcka Fara

\paragraph{Anka} Ankor är små ankor som har händer och går på bakbenen. Väldigt mycket lik de ankor som bor i Ankeborg.
Ankor är små och klena men samtidigt uthålliga och smidiga.

En Anka får automatisk FV 20 (B5) i att simma och +4 FV i egenskapen Smyga

\paragraph{Dvärg} Dvärgar är korta och kraftigt byggda, oftas iförda stora helskägg som de är mycket stolta över.
De älskar sina berg, guld och mithril. Duktiga hantverkare och naturliga på gruvdrift.

En dvärg har perfekt mörkersyn, får extra +5 FV i geologi, som dessutom blir en primär färdighet.

\paragraph{Halvalv} En förälder är en människa och den andra är en alv. Utseendemässigt påminner de mest
om människor, och har en förlängd livstid jämfört mot människor.

En halvalv får +2 FV i egenskaperna Lyssna och Upptäcka Fara.

\paragraph{Halvlängdsman} Gillar sin tobak och sina lugna stunder fylld med mat vid den öppna spisen.
Visar dock upp oerhörd styrka och mod när det verkligen gäller.

En Halvlängdsman får +2 FV i egenskapen Gömma Sig

\paragraph{Halvorcher} En dyster varelse som inte passar ihop med varken människor eller orcher.
Ser mer ut som människor än orcher.

En halvorch får +4 FV i slagsmål

\paragraph{Människa} Absolut vanligaste rasen med råge. Eftersom vi har människor som måttstock
så finns det inget speciellt med människor

\newpage
\section{Rasmodifkationer}
Olika raser har olika modifikationer på grundegenskaperna
\begin{table}[hbp]
  \begin{tabular}{|l|l|l|l|l|l|l|l|}
    \hline
    RAS           & STY     & FYS      & SMI     & INT    & PSY     & KAR     & STO       \\
    \hline
    Anka          & -4      & +2       & +2      & $\pm$0 & $\pm$0  & -3      & 3-6 (5)   \\
    \hline
    Dvärg         & +3      & +2       & $\pm$0  & $\pm$0 & +2      & $\pm$0  & 4-9 (7)   \\
    \hline
    Halvalv       & $\pm$0  & $\pm$0   & +2      & $\pm$0 & $\pm$0  & +1      & 7-16 (12) \\
    \hline
    Halvlängsman  & -4      & +3       & +3      & $\pm$0 & +2      & $\pm$0  & 3-6 (5)   \\
    \hline
    Halvorch      & +2      & +2       & -1      & $\pm$0 & $\pm$0  & -3      & 8-18 (13) \\
    \hline
    Människa      & $\pm$0  & $\pm$0   & $\pm$0  & $\pm$0 & $\pm$0  & $\pm$0  & 8-18 (13) \\
    \hline
    Alv           & -1      & $\pm$0   & +3      & $\pm$0 & $\pm$0  & +2      & 8-14 (11) \\
    \hline
  \end{tabular}
\end{table}

\section{Yrken}

Det finns många olika yrken för en Rollperson, ungefär 11 stycken med olika under-yrken.
Det som skiljer dom åt är vilka sekundära färdigheter de kan räkna som yrkesfärdigheter, och
vilken unik yrkesegenskap de har.

För att kunna tillhöra en viss yrkesgrupp måste rollpersonen uppfylla vissa grundegenskaper, som
står listade vid varje yrke.

Det är bara primära färdigheter och yrkesfärdigheter som kan läras in vid skapandet av Rollpersonen,.

Under spelets gång kan man lära sig olika sekundära färdigheter, men till ett högre pris än vad
yrkesfärdigheterna kostar.

Vid skapandet av rollpersonen får bara 12 yrkesfärdigheter väljas, och för de som vill ha
magikeryrket får bara välja 9 stycken yrkesfärdigheter.

De färdigheter som beskrivs under ``Möjliga yrkesfärdigheter'' och inte väljs vid skapandet av
rollpersonen blir som vanliga sekundära färdigheter.

Vissa färdigheter som Hantverk och Spela instrument har under färdigheter, som det specifika hantverket
eller det specifika musik instrumentet. Varje underfärdighet räknas som en vanlig separat färdighet.

\newpage
\section{Grundegenskaps krav för yrken}
För att tillhöra en viss yrkesgrupp gäller det att grundegenskaperna matchar tabellen

\begin{table}[hbp]
  \begin{tabular}{|l|l|l|l|l|l|l|l|}
    \hline
    YRKE         & STY     & FYS      & SMI     & INT    & PSY     & KAR     & STO  \\
    \hline
    Bard         & ~      & ~         & 12      & ~      & ~       & 14      & ~   \\
    \hline
    Helare       & ~      & ~         & ~       & 12     & 12      & ~       & ~   \\
    \hline
    Krigare      & 14     & 12        & ~       & ~      & ~       & ~       & ~   \\
    \hline
    Lärdman      & ~      & ~         & ~       & 16     & ~       & ~       & ~   \\
    \hline
    Lönn.Mörd,   & ~      & ~         & 14      & ~      & 12      & ~       & ~   \\
    \hline
    Magiker      & ~      & ~         & ~       & 12     & 14      & ~       & ~   \\
    \hline
    Munk         & ~      & ~         & ~       & 12     & 12      & ~       & ~   \\
    \hline
    Sjöfarare    & ~      & 12        & 12      & ~      & ~       & ~       & ~   \\
    \hline
    Riddare      & 14     & 12        & ~       & ~      & 12      & ~       & ~   \\
    \hline
    Tjuv         & ~      & ~         & 16      & ~      & ~       & ~       & ~   \\
    \hline
    Utbygds.Jäg. & ~      & 12        & 12      & ~      & 12      & ~       & ~   \\
    \hline
  \end{tabular}
\end{table}

\newpage
\section{Yrkesfärdigheter}

\paragraph{Bard} Barden kan Spela eller Sjunga under minst en minut och får +5 i Karisma.
Detta ökar C.L. i alla KAR baserade färdigheter med Cl +5. Effekten håller i sig en timme.
Karisma baserade färdigheter är t ex bluffa, övertala, muta, skådespeleri)

\paragraph{Helare} En helare kan göra handpåläggning på en skadad varelse,
varelsen får 1 KP/SR om helaren spenderar lika många PSY poäng.

En helare kan också Fördriva Sjukdom, om helaren kan övervinna sjukdomens
svårighetsgrad. Helaren har PSY/2 mot sjukdomen.

Helaren kan också Neutralisera Gift, om helarens PSY/2 övervinner giftets svårighetsgrad.

Dessa förmågor kostar lika mycket PSY som sjukdomen/giftets styrka, även om helaren
misslyckas.  Det blir därför svårare att Hela flera sjukdomar efter varandra

PSY poäng återvinns på sedvanligt sätt.


\paragraph{Krigare} En krigare får +5 på alla initiativ slag

\paragraph{Lärd man} En lärd man är snusförnuftig via sina djupa kunskaper och får därmed -5
 på alla slag på skräcktabellen

\paragraph{Lönnmördare} En lönnmördare kan smyga upp bakom ett offer och attackera. Detta kräver
både ett lyckat Smyga från lönnmördaren och en misslyckad Upptäcka Fara från offret.

\begin{enumerate}[x]
  \item Ett lyckat slag ger dubbel skada, men ingen Skadebonus
  \item Ett perfekt slag ger fyrdubbel skada, men ingen Skadebonus
  \item Ett fumlat slag ger skada som en vanlig attack, men ingen Skadebonus
\end{enumerate}

En attack bakifrån kan bara göras mot humanoider som inte är 2 meter högre rollpersonen

\paragraph{Magiker} Magiker har ingen speciell yrkesförmåga förutom att kunna lära sig
magi och besvärjelser från början

\paragraph{Munk} En munk kan igenom meditation höja sin FV med 1 i valfri färdighet.
Meditationen tar en SR per ökad FV. En färdighet kan max höjas till FV*2 på detta sätt.
Höjningen håller i sig en minut efter avslutad meditation.

\paragraph{Sjöfarare} En sjöfarare har kraftigt motstånd mot element, som kyla och eld.
Även magisk frambringad element. En Sjöfarare får +5 i FYS, STO och STY när hen ska
motstå olika element

\paragraph{Riddare} En riddare kan använda 5 PSY för att träffa valfri kroppsdel

En riddare kan spendera 5 PSY päng för att göra maximal skada med maximal sin skadebonus.

\paragraph{Tjuv} En tjuv kan spendera upp till 3 PSY poäng för att modifiera CL på valfri
färdighet. Färdigheten kan användas 2 gånger per dag, vila på 8 timmar (4 för alver) sedan
kan färdigheten användas igen.

\paragraph{Utbygdsjägare} En utbygdsjägare kan lära sig magiskolan Animism som yrkesfärdighet.
Utbygdsjägaren kan inte lära sig besvärjelser från start, då rollpersonen skapas.
Utbygdsjägaren kan även lära sig alla allmäna besvärjelser.

En utbygdsjägare kan maximalt lära sig Animism till Skolvärde 12.


\section{Köpa Grundegenskaper}

TODO

\section{Kroppspoäng}
När än rollperson eller en SLP tar skada, så kommer den skadan att dras från kroppsdelen som träffas och från
den totala kroppspoängen.

När kroppspoängen på en kroppsdel blir 0, så blir kroppsdelen obrukbar, om huvudet når noll så svimmar personen.

När totala kroppspoängen blir 0 förlorar man medvetandet och riskerar att förblöda och dö.

Antalet totala Kroppspoäng är (FYS+STO)/2 avrundat uppåt. Kroppsdelarnas KP baseras på den totala KP.

\section{Kroppspoäng tabell}

TODO

\section{Skadebonus}

TODO

\section{Förflyttning}

TODO

\section{Särskilda förmågor}
När du skapar din rollperson så får du slå på en tabell för särskilda förmågor
som gör din rollperson till det lilla extra, unika.

Du måste betala minst 1 BP för att slå på tabellen. För varje BP du spenderar
får du +1 på slaget. Använd en 2T20

\begin{enumerate}[x]
  \item För vanlig rollperson, slå 1 gång på tabellen för särskilda förmågor
  \item För extraordinär rollperson, slå 2 gånger på tabellen för särskilda förmågor
  \item För Hjälte som rollperson, slå 3 gånger på tabellen för särskilda förmågor
\end{enumerate}

TODO: Skriv in tabellen från sidan 25

\section{Svärdshand}
TODO

\section{Socialt Stånd}
TODO

\section{Startkapital}
TODO

\section{Återstående bakgrundspoäng}
Om du har kvar några BP kan du nu gå tillbaka och använda dom för att
förbättra grundegenskaper.

Om du inte gör det, så kan du använda dom för att skaffa lite mer erfarenhet
för dina färdigheter.

Multiplicera dina återstående bakgrundspoäng med 5 och lägg till dom till dina
erfarenhetspoäng (EP), som du nu kan börja köpa yrkesfärdigheter för.

\section{Ålder och grundegenskaper}

När du bestämmer din ålder för rollpersonen förändras dina grundegenskaper. Detta
påverkar \textit{inte}  grundegenskaps gränserna för ditt valda yrke.

TODO

\section{Startfärdigheter}
Dina startfärdigheter är de redan förtryckta primära färdigheterna på rollformuläret.
Samt dina yrkersfärdigheter (12 eller 9 stycken). Resten är sekundära färdigheter.
Magiker kan lära sig besvärjelser fårn början.

\section{Räkna ut baschanser}
Baschans (BC) är det FV du får gratis när du skapar din Rollperson. Baschanser baseras
på grundegenskapsvärdet som hör till färdigheten.

Du får Baschanser i alla primära färdigheter samt alla yrkesfärdigheter.
Du får inga baschanser för besvärjelser, speciella regler gäller för Tala,
Läsa/Skriva modersmål.

\begin{table}[hbp]
  \begin{tabular}{|l|l|}
    \hline
    Värde & Baschans \\
    \hline
    1-3   & 0 \\
    \hline
    4-8   & 1 \\
    \hline
    9-12  & 2 \\
    \hline
    13-16 & 3 \\
    \hline
    17-20 & 4 \\
    \hline
    >20   & 5 \\
    \hline
  \end{tabular}
\end{table}

\section{Färdighetsvärde - Kostnadstabell}
TODO REALLY long table

\section{Skicklighetsväde - Kostnadstabell för besvärjelser}
Kostnader för att bli bättre på besvärjelser baserar sig på skolvärde för den magiskolan
som besvärjelsen tillhör.

Ta skolvärdet för besvärjelsen har och multiplicera dess grundkostnad med antalet nivåer
du vill höja.

Till exempel, För knäcka SV 6 så är grundkostnaden 4 och att höja från 7 till 10 ger 3 i nivå skillnad.
Så kolla upp i tabellen för grundkostnaden för Besvärjelsens Skolvärde och multiplicera
grundkostnaden med nivåskillnaden, Grundkostnad * Differens. Här blir det 4*3 = 12 E.P som det kostar
att förbättra KNÄCKA från 7 till 10.

\begin{table}[hbp]
  \begin{tabular}{|l|l|}
    \hline
    S.V  &   Grundkostnad \\
    \hline
    1-3   & 2 \\
    \hline
    4-6   & 4 \\
    \hline
    7-9   & 6 \\
    \hline
    10-12 & 8 \\
    \hline
    13-15 & 10 \\
    \hline
    16-18 & 12 \\
    \hline
    19-21 & 14 \\
    \hline
    +3    & +2 \\
    \hline
  \end{tabular}
\end{table}

\section{Bakgrund}
Äntligen är rollpersonen färdig, nu är det dags att fundera ut rollpersonens
utseende, historia och beteende. Bakgrunden borde inkludera frisyr som beskriver
hårfärgen och längden. Vilken ögonfärg och eventuell makeup har din RP, muskelös
kropp, smidig eller en liten mage. Ljus eller mörk röst, kommenderande eller vänlig röst.
Vilken klädsel bär du, har du ärr någonstans efter tidigare äventyr.

Låt fantasin flöda.

Bakgrundshistorien borde kunna svara på frågor såsom varifrån du kommer, varför är du ute
och äventyrar. Beskriv ditt stora mål i livet, hur kommer det sig att du är så duktig som du är
på vad det är du nu gör.

Har du familj och vänner någonstans? Har du äventyrar tidigare?

När du spelar din rollperson så ska du agera och tänka som hen, tänk dig in i situationen.

\section{Alternativ och Snabbare sätt att skapa Rollpersoner och SLP}
Istället för att använda Erfarenhetspoäng och Bakgrundspoäng så använder du tärningar för att
få fram resultat.

Du bestämmer själv vilken ras, yrke, kön och ålder du vill vara. Resten är upp till tärningas vilja.
Det finns till och med en chans att din karaktär till och med blir bättre än att använda erfarenhetspoäng.

\paragraph{Grundegenskaper} Alla sju grundegenskaper slås fram. I kapitlet Varelser i grundreglerna så står
alla tärningskombinationer för de olika raserna du kan spela.

Du slår fram 3 st kompletta uppsättningar och bestäm själv vilken av dessa 3 du vill använda.

Inom gruppen du har valt, så får du flytta över poäng från en grundegenskap till en annan för 2 mot 1
basis. Det vill säga för att höja en grundegenskap med 1 poäng måst du sänka en grundegenskap med 2 poäng,
eller två egenskaper med ett poäng vardera.

Tänk på två saker när du väljer tärningsgrupp och flyttar om grundegenskaper, din Rollperson kan inte få
mer än maximala värdet för någon grundegenskap och din rollperson måste uppfylla grundegenskaps villkoret
för det yrket du vill tillhöra.

\paragraph{Kroppspoäng} Kroppspoäng räknas ut som vanligt (FYS+STO)/2 och alla kroppsdelars KP baseras på
detta som vanligt

\paragraph{Särskilda Förmågor} Du får slå på tabellen särskilda förmågor med 2T20, istället för 2T20+BP

\paragraph{Svärdshand} Du får slå med 2T4 (istället för 2T6+BP). Om båda tärningarna visar samma resultat
så får du slå om en av dom och lägga till på resultatet, om även den tärningen visar samma siffra, så
slå repetera.

\paragraph{Socialt stånd} Slå 2T6 (istället för 2T6+BP). Samma tärnings regler som för svärdshand.

\paragraph{Startkapital} Samma metod som för Svärdshand och Socialt stånd. Startkapitalet kan inte
bli högre eller lägre än $\pm$10 än ditt sociala stånd

\paragraph{Start färdigheter} Dina erfarenhetspoäng baseras som vanligt på din ålder. Du får
lägga till 25 EP och därefter måste du dra bort kostnaden för din ras.

Till exempel - En alv har alltid 150 EP (pga av dess icke-varierande ålder). En ung anka har 150+25$\pm$0 = 175
Medelåldersmänniska har 250+25-10=215 EP.

\section{Färdigheter}

\subsection{Chanser att lyckas (CL)}
En rollperson har ett färdighetvärde mellan 1 och 20 i sina olika inlärda färdigheter. Hens CL modifieras
beroende på situation. Det är lättare att göra saker i lugn och ro än att göra dom under stress och livsfara.
Det är även lättare att klättra upp för en repstege än att klättra upp för en hal klippvägg.

Därför läggs modifikationer på ditt FV som blir din nya CL, om man slår under CL så har man lyckats med
sin färdighet

\subsection{Färdighetskategorier - 2 st}

\paragraph{Kategori A} Använder Chanser att Lyckas ligger oftast mellan 1 och 20, slå en T20 för att
se om du har lyckats med din färdighet. Ibland slår SL undangömda slag så att rollpersonen inte vet om
den lyckas eller inte, ibland vet inte ens rollpersonen vad SL slår för.

\paragraph{Kategori B} Slumpen är ofta irrelevant, om de inte är exceptionella undantag. Som till exempel
att rollpersonen inte har sovit på en vecka, eller rollpersonen är påverkad av sinnes slöande droger.

Färdighetsvärden omvandlas till en FV-B skala mellan B0 och B5

\begin{table}[hbp]
  \begin{tabular}{|l|l|}
    \hline
    Färdighet         & Baserad på grundegenskap \\
    \hline
    Läsa/Skriva modersmål & INT(elligens) [special] \\
    Tala modersmål        & INT(elligens) [special] \\
    Finna Dolda Ting  & INT(elligens) \\
    Första hjälpen    & INT(elligens) \\
    Gömma sig         & INT(elligens) \\
    Spåra             & INT(elligens) \\
    Värdera Föremål   & INT(elligens) \\
    Smyga             & SMI(dighet) \\
    Stjäla Föremål    & SMI(dighet) \\
    Hoppa             & SMI(dighet)   \\
    Klättra           & SMI(dighet)   \\
    Rida              & SMI(dighet) \\
    Slagsmål          & STY(rka) \\
    Bluffa            & KAR(isma) \\
    Sjunga            & KAR(isma) \\
    Köpslå            & KAR(isma)  \\
    Övertala          & KAR(isma) \\
    Upptäcka Fara     & PSY(ke) \\
  \end{tabular}
\end{table}

\subsection{Sekundära Färdighetstabellen}

TODO






\end{document}

%%% Local Variables:
%%% mode: latex
%%% TeX-master: t
%%% End:
